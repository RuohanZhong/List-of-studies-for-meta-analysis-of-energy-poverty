\documentclass{article}
\usepackage{hyperref}
\usepackage{xltabular}
\usepackage{ragged2e}
\usepackage{caption}
\usepackage{natbib}



\title{List of previous studies for meta-analysis of consequence of energy poverty}

\author{Ruohan Zhong\\ \href{mailto:ruohanzhong@ruc.edu.cn}{ruohanzhong@ruc.edu.cn}}

\date{June 2025}



\begin{document}


\maketitle

\newpage

\begin{xltabular}{\textwidth}{
    >{\RaggedRight}p{1.4cm} 
    >{\RaggedRight}p{1.7cm}
    >{\noindent\arraybackslash}X
    >{\RaggedRight}p{2cm}
}

\caption{List of previous studies on the consequences}
\label{tab_list_consequence}\\
\hline
\textbf{Period} & \textbf{Country} & \textbf{Results} & \textbf{Source}\\
\hline
\endfirsthead
\hline
\textbf{Period} & \textbf{Country} & \textbf{Results} & \textbf{Source}\\
\hline
\endhead
\hline
\multicolumn{4}{r}{Next page}
\endfoot
\hline
\endlastfoot

\multicolumn{4}{l}{\textbf{\textit{Social system}}}\\
\hline
2008/09, 12/13 & Lao PDR & Energy poverty negatively impacts households’ average school years and health status. & \citet{oum2019energy} \\
2015 & Cambodia & Energy poor households have a higher probability of its members suffering from respiratory problems, spending more on medical care, having a higher dropout rate from schools and lower earning opportunities than the households without energy poverty. & \citet{phoumin2019cambodia} \\
2013 & Spain & Fuel poverty influence the reference indifference curve. & \citet{rodriguez2019fuel} \\
2012, 14, 16 & China & A statistically significant and robust negative impact on health from energy poverty is confirmed. & \citet{zhang2019multidimensional} \\
2005-17 & Australia & Being in fuel poverty lowers subjective wellbeing. & \citet{churchill2020fuel} \\
2004-15 & France & There is a significant causal relationship between fuel poverty and self-assessed health status. & \citet{kahouli2020economic} \\
2010, 15 & Ghana & Energy poverty heightens the chances of being mentally unhealthy. & \citet{lin2020multidimensional} \\
2018 & 11 developing countries in Asia & An empirically significant negative causal relationship was found between the indicators of multidimensional energy poverty and health for women. & \citet{abbas2021health} \\
1990-2018 & 79 countrIes in SAS, SSA, and LCN & Access to both electricity and clean energy improve human development in the aggregated sample. & \citet{acheampong2021does} \\
1990-2017 & 50 developing countries & The lower energy poverty is associated with higher health and education outcomes. & \citet{banerjee2021energy} \\
2012, 14, 16, 18 & China & Being energy poor increases the probability of being an entrepreneur. & \citet{cheng2021energy} \\
2001-17 & Australia & Energy poverty can lower health. & \citet{churchill2021energy} \\
2005-18 & Australia & Energy poverty increases the likelihood of experiencing physical violence. & \citet{hailemariam2021impact} \\
2012 & Ghana & There is a quadratic (U-shaped) relationship between multidimensional energy poverty and social status. & \citet{lin2021does} \\
2008, 11 & Ireland & Children growing up in energy poor homes may be especially vulnerable to health impacts. & \citet{mohan2021young} \\
2012, 14, 16, 18 & China & Energy poverty leads to higher levels of depression. & \citet{nie2021energy} \\
2005, 10, 16 & Bangladesh & Multidimensional energy poverty is negatively associated with the health and education status of households. & \citet{omar2021multidimensional} \\
2000-18 & 175 countries & Energy poverty has a detrimental effect on public health. & \citet{pan2021energy} \\
2001-19 & Australia & Energy poverty is positively associated with obesity. & \citet{prakash2021energy} \\
2005, 11/12 & India & Energy poverty has significant negative effects on children’s health and educational achievements. & \citet{rafi2021multidimensional} \\
2014, 16, 18 & China & Energy poverty reduces children's subjective wellbeing. & \citet{zhang2021energy} \\
2015 & China & Multidimensional energy poverty harms both physical and mental health. & \citet{zhang2021household} \\
2010/12 & UK & There are heterogeneous associations between fuel poverty and wellbeing outcomes. & \citet{davillas2022getting} \\
2018 & China & Multidimensional energy poverty has aggravated depression among older adults, and the effect is greater for older adults with higher depression levels. & \citet{hou2022multidimensional} \\
2011, 13, 15, 18 & China & Energy poverty negatively affects senior citizens’ subjective well-being. & \citet{li2022would} \\
2014, 16, 18 & China &  The impact of energy poverty on cognitive and mental health is significantly negative. & \citet{li2022impact} \\
2003-18 & Pakistan & The energy helpless families have a greater likelihood toward the individuals experiencing respiratory issues, paying more on clinical consideration, dropping out of school and lesser acquiring openings than the household without energy destitution. & \citet{liu2022role} \\
2019 & Nigeria & Energy poverty is a causation of catastrophic health expenditure. & \citet{okorie2022association} \\
2015, 17 & Indonesia & There is a negatively significant impact on education for both energy-poor condition. & \citet{oktaviani2022energy} \\
2014, 16, 18 & China & Both the accessibility and affordability indicators of energy poverty harm the quality of life. & \citet{qin2022impact} \\
2014-18 & Turkey & The household fuel poverty is negatively associated with household happiness. & \citet{ucal2022household} \\
2018 & Indonesia & The low accessibility to electricity leads to a lower health condition. & \citet{utami2022multidimensional} \\
2018 & China & Energy poverty significantly hampers the mental and physical health of Chinese people. & \citet{xu2022assessing} \\
2018 & China & Energy poverty would increase the ratio of respiratory disease, hospitalization as well as the healthcare expenditure. & \citet{zhang2022there} \\
2018 & China & There is a strong positive impact of energy poverty on depression at the upper quantile of depression scores, but no impact at the middle and lower quantiles. & \citet{zhang2022energy} \\
2005–21 & Australia & Being energy poor is associated with a decline in the likelihood of alcohol use. & \citet{amega2023energy} \\
2020 & Kenya & There is a strong, statistically significant impact of energy poverty on human health. & \citet{ang2023analysis} \\
2005, 10, 16 & Bangladesh & The results obtained utilizing a mixed-effect regression approach suggest that energy subsidies and energy poverty are significantly linked, and energy subsidies improve social wellbeing by mediating effects of energy poverty. & \citet{hosan2023evaluating} \\
2021 & Ghana & There is a significant positive impact of energy poverty on functional disability. & \citet{nsenkyire2023energy} \\
2016/17 & Ghana & A standard deviation increase in multidimensional energy poverty reduces child health, education, and cognitive skills by 0.155, 0.13, and 0.402 standard deviation respectively. & \citet{nsenkyire2023household} \\
2017/19 & 15 developing countries in SAS and SSA & Children residing in energy-poor households face a higher risk of experiencing disabilities compared to those in energy non-poor households. & \citet{sen2023evaluating} \\
1990-2020 & Pakistan & The improvements in energy development (characterized by a decline in energy poverty) significantly enhance school enrollments. & \citet{sharif2023unveiling} \\
2000-21 & 98 countries & Access to clean cooking fuels and technologies is associated with improvement in the gender parity index for secondary and tertiary education enrolment. & \citet{acheampong2024energy} \\
2016/17 & Ghana & Energy poverty increases household health expenditures. & \citet{bukari2021energy} \\
2000-20 & 48 countries in SSA & The access to electricity reduce infant, child, and maternal mortality across all quantiles. & \citet{byaro2024tackling} \\
2015 & China & Fertility intentions show a significant inverted U-shaped trend with the increase of multidimensional energy poverty. & \citet{chang2024energy} \\
2020 & France & Being fuel poor decreases the mental health score by 6.3 points out of 100. & \citet{charlier2024fuel} \\
2005-20 & China & Energy poverty has a detrimental impact on local and surrounding areas' social welfare. & \citet{dong2025energy} \\
2000-16 & 143 countries & Reducing energy poverty mitigates health vulnerability. & \citet{fan2024energy} \\
2018/21 & Turkey & Energy poverty is negatively associated with higher health status & \citet{ipek2024energy} \\
2008-17 & South Africa & Energy poverty is associated with an increase in mental distress. & \citet{koomson2024energy} \\
2000-20 & 185 countries & Energy poverty has significantly negative impacts on public health. & \citet{lee2024impact} \\
2018 & China & Households' unclean fuel use significantly damages people's health. & \citet{li2024household} \\
2014 & Pakistan & Households with energy impoverishment have members who are more likely to suffer from respiratory conditions. & \citet{liang2024dynamic} \\
2018 & China & Energy poverty amplifies the preference for boys within households. & \citet{liu2024women} \\
2022 & Kenya & There is a significantly negative impact of energy poverty on self-reported health and a positive effect on health deprivation among women. & \citet{maket2024health} \\
2000-21 & 42 countries in Africa & Energy poverty has a negative and significant effect on total health expenditure and on external health assistance to government. & \citet{ngounou2025does} \\
2011 & Viet Nam & Energy poverty significantly and negatively impacts the life satisfaction of the elderly. & \citet{nguyen2024elderly} \\
2016 & Viet Nam & There is a negative relationship between energy poverty and health expenditure. & \citet{nguyen2024energy} \\
2012, 14, 16, 18 & China & Energy poverty leads to higher levels of total (305 yuan/year), out-of-pocket (199 yuan/year), inpatient (230 yuan/year) and other (113 yuan/year) health care expenditures. & \citet{nie2024does} \\
2019/20 & Gambia, Sierra Leone, and Ghana & There is a significant negative relationship between multidimensional energy poverty and women's subjective well-being and cognitive health. & \citet{nsenkyire2024multidimensional} \\
2014 & Ghana & An 8.2–8.7\% increase in the likelihood of female youth engaging in risky sexual activities for each percentage increase in energy poverty. & \citet{okyere2024energy} \\
2000-20 & 34 countries in SSA & A negative and significant effect of access to electricity and access to clean energy for cooking on respiratory disease in sub-Saharan Africa. & \citet{pondie2024energy} \\
2022 & China & Farmers' physical and mental health is significantly negatively affected by multidimensional energy poverty. & \citet{qin2024can} \\
2000-19 & 52 countries & There is a robust association between energy poverty and an increased likelihood of anxiety and depression among school-aged children. & \citet{sen2024unveiling} \\
2003-21 & Pakistan & Energy-poor households have been shown to be more prone to experience respiratory illnesses, pay more for health care, drop out of school, and have less employment opportunities. & \citet{shabbir2024energy} \\
2003-20 & China & Energy poverty has a significant negative impact on social welfare. & \citet{tang2024environmental} \\
2012, 14, 16, 18 & China & Energy poverty influences knowledge acquisition by impacting the time allocated, health and fundamental competencies. & \citet{wang2024has} \\
2018 & China & Energy poverty exerts a significant negative impact on the health and welfare of middle-aged and older adults. & \citet{wang2024impact} \\


\hline
\multicolumn{4}{l}{\textbf{\textit{Economic system}}}\\
\hline
2011, 14 & Indonesia & The access to electricity and modern cooking fuel reduced the rate of malnutrition in the village. & \citet{sambodo2019state} \\
1995-2007 & 7 countries in SAS & Energy poverty has a negative influence on economic development in both the long-run and the short-run. & \citet{amin2020does}\\
2018/20 & UK & There is a statistically robust relationship between fuel poverty indicators and self-reported measures of current financial distress, with stronger effects for subjective indicators. & \citet{burlinson2021fuel} \\
2018/19 & Pakistan & Energy poverty has a positive and significant impact on health poverty. & \citet{nawaz2021energy} \\
2002-17 & 45 developing countries & There are significant negative impacts of no access to clean fuels and technologies for cooking and the no access to electricity on total factor productivity. & \citet{nguyen2021vicious} \\
1995-2014 & Indonesia & In long-run, the increment of 1\% for both energy consumption will result -1.12\% effect towards inflation. & \citet{wan2021energy} \\
2015 & China & Both extensive energy poverty and intensive energy poverty have a significant negative effect on the wages of rural workers. & \citet{wu2021does} \\
2008-17 & China & There is a negative correlation between energy poverty and the energy efficiency of the construction industry. & \citet{zhang2021study} \\
2002-15 & 73 countries &  Energy poverty appears to have an increased influence on economic vulnerability. & \citet{nguyen2022nexus} \\
1990–2018 & BRICS countries & The access to electricity play a considerable role in promoting economic development. & \citet{raghutla2022energy} \\
2002-19 & China & Energy poverty can significantly lower agricultural technical efficiency. & \citet{shi2022impact} \\
2014, 16, 18 & China & Energy poverty reduced irrigation water efficiency by 28.3 - 42.4\%. & \citet{shi2022understanding} \\
2014, 16, 18 & China & Energy poverty has a negative effect on Internet use. & \citet{wang2022impact} \\
2005-20 & 35 countries in SSA & Aggregate energy poverty index has a significant positive influence on agricultural productivity. & \citet{dimnwobi2023energy} \\
2000-17 & 100 developing countries & Energy poverty negatively affects productive efficiency. & \citet{ndubuisi2023too} \\
1995-2018 & 33 OECD countries & The estimation shows a robust association among energy poverty, green innovation, and economic complexity with improving economic performance. & \citet{zhang2023role} \\
2014 & Burkina Faso & An increase in multidimensional energy deprivation is associated with a decrease in women’s economic well-being. & \citet{compaore2024energy} \\
2000-21 & 44 countries in Africa & Reducing energy poverty is a necessary condition for the industrialization of African countries. & \citet{djeunankan2024linking} \\
2020 & China & Energy poverty in rural households tends to increase the number of inclusive loans. & \citet{gu2024dynamic} \\
2016/18 & Nepal & Multidimensional energy poverty significantly reduces both household consumption and savings, with more pronounced effects on savings. & \citet{koirala2024multidimensional} \\
2002-17 & China & Energy poverty may greatly reduce agricultural water efficiency. & \citet{shi2024influence} \\
2012, 14, 16, 18 & China & Energy poverty significantly reduces household development resilience. & \citet{yang2024energy} \\
2011-20 & China & The impact of energy poverty on improving energy-related efficiency is significantly negative. & \citet{zhang2024improving} \\


\hline
\multicolumn{4}{l}{\textbf{\textit{Environmental system}}}\\
\hline
2002-17 & China & Energy poverty can accelerate the growth of CO$_2$ emissions. & \citet{zhao2021assessing} \\
2004-16 & China & The carbon intensity of the construction industry increases by 1.683 units per unit increase of energy poverty, showing a positive impact. & \citet{zhang2022empirical} \\
2000-17 & China &  Rural energy poverty alleviation exerts an inverted U-shaped effect on agricultural carbon emissions. & \citet{li2023does} \\
2006-17 & 18 developing countries in Asia & Energy poverty positively affects carbon emissions growth. & \citet{yahong2023clean} \\
2001-20 & 48 countries participating in BRI & The presence of energy poverty in rural areas is associated with higher levels of carbon emissions. & \citet{xu2023urbanization} \\
2000-19 & 7 emerging (E7) countries & The causality analysis entails that a bidirectional exists between energy poverty, income, and ecological footprint. & \citet{yasmeen2023economic} \\
2002-17 & China & Energy poverty significantly leads to a higher level of carbon lock-in. & \citet{zhao2023does} \\
2019, 21, 22 & Spain & Being energy poor increases the probability to use carbon-intensive energy sources for heating compared to electricity. & \citet{burguillo2024does} \\
2000-18 & 28 countries in Africa & The access to clean energy and electricity negatively impacts the ecological footprint across all the quantiles. & \citet{dada2024moderating} \\
1990-2021 & Nigeria & There is a negative effect of energy poverty on load capacity factor. & \citet{ezenekwe2024balancing} \\
2000–21 & 43 countries in SSA & Access to electricity and clean energy for cooking have a positive and significant effect on deforestation and carbon emissions. & \citet{pondie2024poverty} \\
2000-14 & Brazil, Russia, India, China, and South Africa & There is a positive and significant impact of energy poverty on aggregate material footprint and its components. & \citet{villanthenkodath2024impact} \\
1990-2019 & 193 countries & The role of energy poverty on carbon emissions per capita is also affected by income inequality. & \citet{wang2024interaction} \\
2010-20 & 70 BRI countries & Alleviating energy poverty in BRI countries will lead to an increase in per capita carbon emissions. & \citet{wang2024impact2} \\
2003-19 & 65 countries & Eradicating energy poverty through access to electricity does not reduce CO$_2$. & \citet{zhao2024does} \\


\hline
\multicolumn{4}{l}{\textbf{\textit{Comprehensive system}}}\\
\hline
2002-14 & 51 countries & There is significant evidence of Granger causality between energy poverty and income inequality. & \citet{nguyen2021inquiry} \\
2004-17 & China & Energy poverty eradication can effectively promote the country's green growth. & \citet{zhao2022does} \\
2006-19 & China & Eradicating energy poverty has a significant positive effect on common prosperity. & \citet{dong2023does} \\
2000-20 & 36 countries in SSA & The energy development index, access to electricity, and access to clean energy for cooking positively influence food security. & \citet{pondie2023does} \\
2000-19 & 77 countries & The access to electricity contributes to reduce income inequality, while access to clean fuels and renewable energy development can increase income inequality. & \citet{song2023study} \\
2001-20 & China & Energy poverty significantly and negatively affects internal immigration. & \citet{zhuo2023energy} \\
1990-2021 & Ghana & Income inequality is Granger causally related to access to electricity and rural area access to electricity. & \citet{opoku2024focus} \\
2002-21 & 43 countries in SSA & Access to electricity is an important factor that enhances HDI. & \citet{etudaiye2024fintech} \\
2000–20 & 54 countries in Africa &  Energy poverty (measured by access to electricity and access to clean energy for cooking) increases urbanization in Africa. & \citet{song2024urbanization} \\
2000-21 & 21 countries in LAC & Energy poverty contributes to higher levels of income inequality. & \citet{soto2025energy} \\
2000-17 & China & The eradication of rural energy poverty yields tangible benefits for agricultural sustainable development. & \citet{wang2024does} \\
2007-17 & China & Energy poverty alleviation drives the economic growth of China's eastern region and western region, has an effect on reducing the urban-rural income inequality, and has a significant effect on promoting economic HQD. & \citet{yang2024effect} \\

\end{xltabular}




\clearpage

\bibliographystyle{apalike} 
\bibliography{consequence}

\end{document}
